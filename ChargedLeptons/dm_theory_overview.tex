
% test for first committ

%\subsubsection{Dipole Moments}
%\label{cl:dipole-moments}

%%%%%%%%%%%%%%%%%%%%%%%%%%%%%%%%%%%
% For all our labels we use :cl: (cl for charged leptons), like
% eq:cl:mueg
% That way we will not have problems when combining it

%%%%%%%%%%%%%%%%%%%%%%%%%%%%%%%%%%%%%%%%%%%%%%%%%%%%%%%%%%%
%%%%%%%%%%%%%%%%%%%%%%%%%%%%%%%%%%%%%%%%%%%%%%%%%%%%%%%%%%%
%%%%%%%%%%%%%%%%%%%%%%%%%%%%%%%%%%%%%%%%%%%%%%%%%%%%%%%%%%%
%%%%%%%%%%%%%%%%%%%%%%%%%%%%%%%%%%%%%%%%%%%%%%%%%%%%%%%%%%%


The muon provides a unique opportunity to explore the properties of a
second-generation particle  with great precision. Several muon properties make
these measurements possible.  It has a long
lifetime of $\simeq 2.2~\mu$s,  it is produced in the weak decay
$\pi^- \rightarrow \mu^- \bar \nu_\mu$ providing copious numbers of
polarized muons, and the weak decay
$\mu^- \rightarrow e^- \nu_\mu \bar \nu_e $ is self-analyzing providing information on the muon spin direction at the time of decay.

In his famous paper on the relativistic theory of the electron,
Dirac\cite{Dirac28} obtained the correct magnetic moment for the
electron, and he also mentioned the possibility of an electric
 dipole moment, which like the magnetic dipole moment,
would be directed along the electron spin direction.
  The magnetic dipole (MDM) and electric dipole (EDM)
moments are given by
\begin{equation}
\vec \mu = g \left( \frac{Qe}{ 2m}\right) \vec s\, , \qquad
 \vec d = \eta  \left(\frac {Qe  }{ 2
     mc}\right)
\vec s \, ,
\label{eq:MDM-EDMdef}
\end{equation}
where  $Q =  \pm 1$ and $e>0$. Dirac theory predicts $g \equiv 2$,
but radiative corrections dominated by the
lowest-order (mass-independent) Schwinger contribution $a_{e,\mu,\tau} =
 \alpha/(2\pi)$~\cite{Schwinger48} make it necessary to
write the magnetic moment as
\begin{equation}
\mu = \left(1 + a\right)\frac{Qe \hbar }{ 2m}\quad {\rm with} \quad
a = \frac{{g - 2} }{ 2}.
\label{eq:muon-MDM}
\end{equation}

The muon played an important role in our discovery of the generation
structure of the Standard Model (SM) when
 experiments at the Nevis
cyclotron
showed  that $g_\mu$ was consistent with 2~\cite{Garwin57}.
Subsequent experiments at Nevis and CERN showed  that
$a_\mu \simeq \alpha/(2\pi)$~\cite{Garwin60,Charpak61},
implying that in a magnetic field, the muon
behaves like a heavy electron.
The SM value of the muon anomaly is now known
to better than half a part per million (ppm), and has
been measured to a similar precision~\cite{Bennett06}.

The quantity $\eta$ in Eq.~\ref{eq:MDM-EDMdef}
 is analogous to the $g$-value for the magnetic dipole
moment. An EDM violates both {\sl P} and {\sl T}
symmetries~\cite{Purcell50,Landau57,Ramsey58}, and since $C$ is conserved,  ${ C\!P}$ is violated as well.  Thus
searches for EDMs provide an important tool in our quest to
find non-Standard Model ${ C\!P}$ violation.

The measured value of the muon anomalous magnetic moment is in apparent
disagreement with the expected value based on the
SM.  The BNL E821 experiment finds~\cite{hep-ex/0602035}
\begin{equation} \hspace*{-23pt}
    a_\mu(\textrm{Expt}) = 116\,592\,089(54)(33)\times10^{-11},
    \label{eq:e821}
\end{equation}
where $a_\mu=(g-2)/2$ is the muon anomaly, and the uncertainties are
statistical and systematic, respectively.  This can be compared with
the SM prediction~\cite{arXiv:1010.4180,931465}
\begin{equation}
    a_\mu(\textrm{SM})   = 116\,591\,802(42)(26)(02)\times10^{-11},
    \label{eq:SM}
\end{equation}
where the uncertainties are from the $\mathrm{O}(\alpha^2)$ hadronic vacuum
polarization (HVP) contribution, $\mathrm{O}(\alpha^3)$ hadronic
contributions (including hadronic light-by-light (HLbL) scattering),
and all others (pure QED, including a 5-loop
estimate~\cite{arXiv:1110.2826}, and electroweak, including
2-loops~\cite{hep-ph/0212229}). The hadronic contributions dominate
the uncertainty in $a_\mu(\rm SM)$.  The discrepancy between the
measurement and the SM stands at
\begin{equation}
\Delta a_\mu=287(80)\times 10^{-11}
\end{equation}
(3.6 standard deviations ($\sigma$)), when based on the $e^+e^-\to\rm
hadrons$ analysis for the HVP
contribution~\cite{arXiv:1010.4180}. When the HVP analysis is
complemented by $\tau\to\rm hadrons$, the discrepancy is reduced to
2.4$\sigma$~\cite{arXiv:1010.4180}. However, a recent re-analysis,
employing effective field theory techniques, of the $\tau$
data~\cite{arXiv:1101.2872} shows virtual agreement with the
$e^+e^-$-based analysis, which would solidify the current discrepancy
at the 3.6$\sigma$ level. $\Delta a_\mu$ is large, roughly two times
the EW contribution~\cite{hep-ph/0212229}, indicating potentially
large new physics contributions.



%%% BSM


The anomalous magnetic moment of the muon is sensitive to
contributions from a wide range of physics beyond the standard
model. It  will continue to place stringent restrictions on all of
the models, both present and yet to be written down. If  
physics beyond the standard model is discovered at the LHC 
or other experiments,
$a_\mu$  will constitute an indispensable tool to discriminate
between very different types of new physics, especially since it is
highly sensitive to parameters which are difficult to measure at the
LHC. If no new phenomena are found elsewhere, then it represents one of the few ways
to probe physics beyond the standard model. In either case, it will play an
essential and complementary role in the quest to understand physics
beyond the standard model at the TeV scale. 

The muon magnetic moment has a special role because it is
sensitive to a large class of models related and unrelated to electroweak symmetry breaking and
because it combines several properties in a unique way: it is a
flavor- and CP-conserving, chirality-flipping and loop-induced 
quantity. In contrast, many high-energy collider observables at the
LHC and a future linear collider are chirality-conserving, and many
other low-energy precision observables are CP- or
flavor-violating. These unique properties might be the reason why the
muon $(g-2)$ is the only among the mentioned observables which shows a 
significant deviation between the experimental value and the SM
prediction.  Furthermore, while $g-2$ is sensitive
to leptonic couplings, 
$b$- or $K$-physics more naturally probe the hadronic couplings of new
physics. If charged lepton-flavor violation exists, observables such
as $\mu\to e$ conversion can only determine a combination of the
strength of lepton-flavor violation and the mass scale of new
physics. In that case, $g-2$ can help to disentangle the nature of the
new physics. 


((( I would like to reduce this entire thing below to one table.  BCKC)))))

Unravelling the existence and the properties of such new physics
requires experimental information complementary to the LHC.
The muon $(g-2)$, together
with searches for charged lepton flavor violation, electric dipole
moments, and rare decays, belongs to a class of complementary
low-energy experiments.

In fact, 
The role of $g-2$ as a discriminator between very different standard
model extensions is well illustrated by a relation stressed by
Czarnecki and Marciano~\cite{czmar}. It holds in a wide range of
models as a result of the chirality-flipping nature of both  $g-2$ and
the muon mass: If a new
physics model with a mass scale $\Lambda$
contributes to the muon mass $\delta m_\mu(\mbox{N.P.})$, it also
contributes to $a_\mu$, and the two contributions are related as
\begin{equation}
\label{CzMbound} a_\mu(\mbox{N.P.})={\cal O}(1)\times
\left(\frac{m_\mu}{\Lambda}\right)^2 \times \left(\frac{\delta
m_\mu(\mbox{N.P.})}{m_\mu}\right). 
\end{equation}


The ratio $C(\mbox{N.P.})\equiv\delta m_\mu(\mbox{N.P.})/{m_\mu}$
cannot be larger than unity unless there is fine-tuning in the muon
mass. Hence a first consequence of this relation is that new physics
can explain the currently observed deviation (\ref{eq:Delta}) only if
$\Lambda$ is at the few-TeV scale or smaller.

In many models, the ratio $C$ arises from one- or even two-loop
diagrams, and is then suppressed by factors like  $\alpha/4\pi$ or
$(\alpha/4\pi)^2$. Hence, even for a given $\Lambda$, the
contributions to $a_\mu$ are highly model dependent.

It is instructive to classify new physics models as follows:
\begin{itemize}
\item Models with $C(\mbox{N.P.})\simeq1$: Such models are of interest
  since the muon
  mass is essentially generated by radiative effects  at some
scale $\Lambda$.
A variety of such models  have been discussed in~\cite{czmar}, including
extended technicolor or generic models with naturally vanishing bare
muon mass. For examples of radiative muon mass generation within
supersymmetry, see e.g.\ 
\cite{Borzumati:1999sp,Crivellin:2010ty}.  In these models the
new physics contribution to $a_\mu$ can be very large, 
\begin{equation} 
a_{\mu}
(\Lambda) \simeq {m^2_{\mu} \over \Lambda^2}\simeq
1100\times10^{-11}\left(\frac{1\mbox{ TeV}}{\Lambda}\right)^2. 
\end{equation}
and the difference Eq.~(\ref{eq:Delta}) can  be used to place a lower
limit on the new physics mass scale, which is in the few TeV
range~\cite{elp,Crivellin:2010ty}.
\item Models with $C(\mbox{N.P.})={\cal O}(\alpha/4\pi)$:
Such a loop suppression happens in many models with new weakly
interacting particles like $Z'$ or $W'$, little Higgs or certain extra
dimension models.  As examples, the contributions to $a_\mu$ in a
model with $\delta=1$ (or
2) universal extra dimensions (UED)~\cite{AppelqDob} and the Littlest Higgs
model with T-parity (LHT)~\cite{Blanke:2007db} are given by
% , other measurements already imply a
%lower bound of 300 (or 500) GeV on the masses of the extra states,
%and the one-loop contributions to  a_\mu\ are correspondingly small,
%\begin{equation}
 %a_\mu(\mbox{UED})&\simeq&-5.8\times10^{-11}(1+1.2\delta)S_{\rm
 % KK},\\
% a_\mu(\mbox{LHT})&<& 12\times10^{-11}
%\end{equation}
 with $|S_{\rm KK}| _{\sim}^{<}1$~\cite{AppelqDob}.
A difference as large as
Eq.~(\ref{eq:Delta}) is very hard to accommodate unless the mass scale
is very small, of the order of $M_Z$, which however is often excluded
e.g.\ by LEP measurements.
So typically these models predict very small contributions to $a_\mu$
and will be disfavored if the current deviation will be confirmed by
the new $a_\mu$ measurement.

Exceptions are provided by models where new particles
interact with muons but are otherwise hidden from searches. An example
is the model with a new gauge boson associated to a gauged lepton
number $L_\mu-L_\tau$ \cite{LmuLtau}, where a gauge boson mass of
${\cal O}(100\mbox{ GeV})$ and large $a_\mu$ are viable.
\item Models with intermediate values for $C(\mbox{N.P.})$ and mass
  scales around the weak scale: In such
  models, contributions to $a_\mu$ could be as large as
  Eq.~(\ref{eq:Delta}) or even larger, or smaller, depending on the
  details of the model. This implies that a more precise
  $a_\mu$-measurement will have significant impact on such models and
  can even be used to measure model parameters. Supersymmetric (SUSY) models
  are the best known examples, so muon $g-2$ would have substantial
  sensitivity to
 SUSY particles.
Compared to generic perturbative models,
supersymmetry provides an enhancement to $C(\mbox{SUSY})={\cal
  O}(\tan\beta\times\alpha/4\pi)$
and to $ a_\mu(\mbox{SUSY})$ by a factor $\tan\beta$ (the ratio of
the vacuum expectation values of the two Higgs fields). Typical SUSY
diagrams for the magnetic dipole moment, the electric dipole moment,
and the lepton-number violating conversion process $\mu \rightarrow
e$ in the field of a nucleus are shown pictorially in
Fig.~\ref{fg:susy}. The shown diagrams contain the SUSY partners of
the muon, electron and the SM U(1)$_Y$ gauge boson, $\tilde{\mu}$,
$\tilde{e}$, $\tilde{B}$. The full SUSY contributions involve also the
SUSY partners to the neutrinos and all SM gauge and Higgs bosons. In a
model with SUSY masses equal to $\Lambda$ 
the SUSY contribution to $a_{\mu}$ is given
by~\cite{czmar} \begin{equation} \label{amususy}
%&\simeq& {\alpha(M_Z) \over 8 \pi \sin^2 \theta_W} {m^2_{\mu} \over \tilde
%    m^2}
%\tan \beta \left( 1 - {4\alpha \over \pi}\ln {\tilde m \over m_{\mu}}\right)
%  \\
 a_{\mu}({\rm SUSY})\, \simeq \,{\rm sgn}\, (\mu) \ 130 \times 10^{-11}\ \tan \beta\
\left({100\ {\rm GeV}  \over \Lambda}\right)^2
%\\
% &\simeq& \ {1.31}\ \  {\rm ppm }  \ \ \  \tan \beta\
%  \left({100\ {\rm GeV} \over \tilde m}\right)^2 \\
\end{equation}
which indicates the dependence on $\tan \beta$,
and the SUSY mass scale,  as well as the sign of the
SUSY $\mu$-parameter. The formula still approximately applies even if
only the smuon and chargino masses are of the order $\Lambda$
but e.g.\ squarks and gluinos are much heavier. However the SUSY
contributions to $a_\mu$ depend strongly on the details of mass
splittings between the weakly interacting SUSY particles.
Thus muon $g-2$ is sensitive to  SUSY models with SUSY masses
in the few hundred GeV range, and it will help to measure SUSY
parameters. 

There are also non-supersymmetric models with similar
enhancements. For instance, lepton flavor mixing can help. An example
is provided in Ref.\ \cite{BarShalom:2011bb} by a model with two Higgs
doublets and four generations, which can accommodate large
$\Delta a_\mu$ without violating constraints on lepton flavor
violation. In variants of Randall-Sundrum models
\cite{Davoudiasl:2000my,Park:2001uc,Kim:2001rc} and large
extra dimension models \cite{Graesser:1999yg}, large
contributions to $a_\mu$ might be possible from exchange of
Kaluza-Klein gravitons, but the theoretical evaluation
is difficult because of cutoff dependences. A recent evaluation of the
non-graviton contributions in Randall-Sundrum models, however,
obtained a very small result \cite{Beneke:2012ie}. 



Further examples
include scenarios of unparticle physics
\cite{Cheung:2007zza,Conley:2008jg} (here a more precise
$a_\mu$-measurement would constrain the unparticle scale dimension and
effective couplings), generic models with a hidden sector at the weak
scale \cite{McKeen:2009ny} or a model with the discrete flavor 
symmetry group $T'$ and Higgs triplets \cite{Ho:2010yp} (here
a more precise $a_\mu$-measurement would constrain hidden sector/Higgs
triplet masses and couplings), or the model proposed in
Ref.\ \cite{Hambye:2006zn}, which implements the idea that neutrino
masses, leptogenesis and the deviation in $a_\mu$ all originate from
dark matter particles. In the latter model, new leptons and scalar
particles are predicted, and $a_\mu$ provides significant constraints
on the masses and Yukawa couplings of the new particles.
\end{itemize}

%
The following types of new physics scenarios are quite different from
the ones above:
\begin{itemize}
\item Models with extended Higgs sector but without the
  $\tan\beta$-enhancement of SUSY models. Among these models are the
  usual two-Higgs-doublet models. The one-loop contribution
  of the extra Higgs states to $a_\mu$ is suppressed by two additional powers of
  the muon Yukawa coupling, corresponding to $a_\mu(\mbox{N.P.})\propto
  m_\mu^4/\Lambda^4$ at the one-loop level. Two-loop effects from
  Barr-Zee diagrams can be larger \cite{Krawczyk:2002df}, but typically the
  contributions to   $a_\mu$ are negligible in these models.
\item Models with additional light particles with masses below the
  GeV-scale, generically called dark sector models: Examples are
  provided by the  models of   Refs.\
  \cite{Pospelov:2008zw,Davoudiasl:2012qa}, where additional light
  neutral gauge bosons can affect electromagnetic interactions. Such
  models are intriguing since 
  they completely decouple $g-2$ from the physics of EWSB, and since
  they are hidden from collider searches at LEP or LHC (see however
  Refs.\ \cite{Essig:2009nc,Davoudiasl:2012ig} for studies of possible
  effects at dedicated low-energy colliders and in Higgs decays at the
  LHC). They can lead to
  contributions to $a_\mu$ which are of the same order as the deviation
  in Eq.~(\ref{eq:Delta}). Hence the new $g-2$ measurement will provide
  an important test of such models.
\end{itemize}
To summarize:
many well-motivated models can accommodate larger contributions to
$a_\mu$ --- if any of these are realized $g-2$ can be used to constrain
model parameters; many well-motivated new physics models
give tiny contributions to $a_\mu$ and would be disfavored if the
more precise $g-2$ measurement confirms the current deviation. There are also examples of models which lead to
similar LHC signatures but which can be distinguished using $g-2$.


