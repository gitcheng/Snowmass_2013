
%%%%%%%%%%%%%%%%%%%%%%%%%%%%%%%%%%%
% For all our labels we use :cl: (cl for charged leptons), like
% eq:cl:mueg
% That way we will not have problems when combining it

%%%%%%%%%%%%%%%%%%%%%%%%%%%%%%%%%%%%%%%%%%%%%%%%%%%%%%%%%%%
%%%%%%%%%%%%%%%%%%%%%%%%%%%%%%%%%%%%%%%%%%%%%%%%%%%%%%%%%%%
%%%%%%%%%%%%%%%%%%%%%%%%%%%%%%%%%%%%%%%%%%%%%%%%%%%%%%%%%%%
%%%%%%%%%%%%%%%%%%%%%%%%%%%%%%%%%%%%%%%%%%%%%%%%%%%%%%%%%%%



At the magic momentum, the equation for the spin procession frequency of a charged particle in a storage ring is given by
\begin{equation}
\vec{\omega}_{a\eta}= \vec \omega_a + \vec \omega_\eta = -
 a \frac {Qe}  {m}
 \vec{B}
 -  \eta \frac {Qe}{2m}
 \left[ \frac {\vec{E}} {c}  +  \vec{\beta} \times \vec{B} \right] .
 \label{eq:omegaa-edm1}
\end{equation}
The precession frequency is by far dominated by the $a= (g-2)/2$ term.  
The key to extracting sensitivity to the EDM term $\eta$ is to find ways of reducing or eliminating the motion due to the magnetic term $a$. 

The first method is to use a magnetic storage ring such as the Muon g-2 experiment to extract a 
limit on the muon EDM.  In the muon rest frame, the muon sees a strong motional electric field
 pointing towards the center of the ring adding a small horizontal component to the precession 
 frequency vector that tilts the rotation plane.  For a positive EDM, when the spin is pointing into the 
 ring it will have a negative vertical component and when the spin is pointing to the outside of the ring 
 it will have a positive vertical component.  Since the positrons are emitted along the spin direction, 
 this asymmetry maps into the positron decay angle. Since the asymmetry is maximized when the spin 
 are perpendicular, the angular asymmetry is 90 degrees out of phase with the $g-2$ precession frequency.  
Searches for this asymmetry have been used to set limits on the muon EDM both at the CERN and Brookhaven $g-2$ experiments. 

 A number of the E989 detector stations will
be instrumented with straw chambers to measure the decay positron
tracks. With this instrumentation, a simultaneous EDM
measurement can be made during the $a_\mu$ data collection,
improving on the  E821 muon EDM~\cite{Bennett08-edm}
 limit by up to two orders of magnitude down to
$\sim 10^{-21}\,  e \cdot {\rm cm}$.   The J-PARC muon $g-2$ proposal also
 will have decay angle information for all tracks and expects a similar improvement.

To go beyond this level for the muon, will require a dedicated EDM experiment that
uses
the ``frozen spin'' method~\cite{Farley04,Roberts2010}.
 The idea is to operate a
muon storage ring off of the  $g-2$ momentum and to use a radial electric field
to cancel the $\omega_a$ term in Eq.~\ref{eq:omegaa-edm1},
 the $g-2$ precession.  The  electric field needed to freeze the spin is
$E \simeq aBc\beta\gamma^2$.
Once the spin is frozen, the EDM will cause a steadily increasing
out-of-plane motion of the spin vector. One stores polarized muons in a ring
with detectors above and below the storage region and forms the asymmetry
(up - down)/(up + down).  To reach a sensitivity of $10^{-24}e \cdot {\rm cm}$ would
require $\sim 4 \times 10^{16}$ recorded events~\cite{Farley04}.
 Preliminary discussions have begun on a frozen spin experiment
using the ~1000 kW beam power available at the Project X 3 GeV rare process campus.

An alternative method is to remove the $g-2$ precession frequency completely by removing 
the magnet and using an electrostatic storage ring. This still requires the particle to be at the 
magic momentum to cancel the motional magnetic field.  For these experiments, counter rotating 
beams are used to cancel the dominant systematic effects associated with stray magnetic fields.  
This idea has been studied in detail for the proton and deuteron with projected sensitivities approaching $10^{-30}$ 
using fairly large storage rings an proton momenta of $700$ MeV.

For the electron, the magic momentum is $15$ MeV.  The smaller momentum would allow for a
 much smaller storage ring.  Initial studies indicate that sensitivities up to $10^{-27}$ $\cdot $cm 
 can be achieved which would be competitive with current limits and would be the best limits for a
  bare fermion.  Furthermore, this would act as a much smaller test bed for the proton storage ring EDM experiment and would 
  help demonstrate that the systematic uncertainties could be controlled.
  
((((((((Add a few paragraphs about the details of the electron proposal))))))))


